%%%%%%%%%%%%%%%%%%%%%%%%%%%%%%%%%%%%%%%%%
% Beamer Presentation
% LaTeX Template
% Version 1.0 (10/11/12)
%
% This template has been downloaded from:
% http://www.LaTeXTemplates.com
%
% License:
% CC BY-NC-SA 3.0 (http://creativecommons.org/licenses/by-nc-sa/3.0/)
%
%%%%%%%%%%%%%%%%%%%%%%%%%%%%%%%%%%%%%%%%%

%----------------------------------------------------------------------------------------
%	PACKAGES AND THEMES
%----------------------------------------------------------------------------------------

\documentclass{beamer}

\mode<presentation> {

% The Beamer class comes with a number of default slide themes
% which change the colors and layouts of slides. Below this is a list
% of all the themes, uncomment each in turn to see what they look like.

%\usetheme{default}
%\usetheme{AnnArbor}
%\usetheme{Antibes}
%\usetheme{Bergen}
%\usetheme{Berkeley}
%\usetheme{Berlin}
%\usetheme{Boadilla}
%\usetheme{CambridgeUS}
%\usetheme{Copenhagen}
%\usetheme{Darmstadt}
%\usetheme{Dresden}
%\usetheme{Frankfurt}
%\usetheme{Goettingen}
%\usetheme{Hannover}
%\usetheme{Ilmenau}
%\usetheme{JuanLesPins}
%\usetheme{Luebeck}
\usetheme{Madrid}
%\usetheme{Malmoe}
%\usetheme{Marburg}
%\usetheme{Montpellier}
%\usetheme{PaloAlto}
%\usetheme{Pittsburgh}
%\usetheme{Rochester}
%\usetheme{Singapore}
%\usetheme{Szeged}
%\usetheme{Warsaw}

% As well as themes, the Beamer class has a number of color themes
% for any slide theme. Uncomment each of these in turn to see how it
% changes the colors of your current slide theme.

%\usecolortheme{albatross}
%\usecolortheme{beaver}
%\usecolortheme{beetle}
\usecolortheme{crane}
%\usecolortheme{dolphin}
%\usecolortheme{dove}
%\usecolortheme{fly}
%\usecolortheme{lily}
%\usecolortheme{orchid}
%\usecolortheme{rose}
%\usecolortheme{seagull}
%\usecolortheme{seahorse}
%\usecolortheme{whale}
%\usecolortheme{wolverine}

%\setbeamertemplate{footline} % To remove the footer line in all slides uncomment this line
%\setbeamertemplate{footline}[page number] % To replace the footer line in all slides with a simple slide count uncomment this line

%\setbeamertemplate{navigation symbols}{} % To remove the navigation symbols from the bottom of all slides uncomment this line
}

\usepackage{graphicx} % Allows including images
\usepackage{booktabs} % Allows the use of \toprule, \midrule and \bottomrule in tables

%----------------------------------------------------------------------------------------
%	TITLE PAGE
%----------------------------------------------------------------------------------------

\title[Bullet Noob and the Quest for the Holy Grail]{Bullet Tutorials} % The short title appears at the bottom of every slide, the full title is only on the title page

\author{Brigham Keys} % Your name
\institute[Collective Tyranny] % Your institution as it will appear on the bottom of every slide, may be shorthand to save space
{
Collective Tyranny
\medskip
\textit{bkeys@bkeys.org} % Your email address
}
\date{\today} % Date, can be changed to a custom date

\begin{document}

\begin{frame}
\titlepage % Print the title page as the first slide
\end{frame}

%----------------------------------------------------------------------------------------
%	PRESENTATION SLIDES
%----------------------------------------------------------------------------------------

%------------------------------------------------

\begin{frame}
\frametitle{What you need}
\begin{itemize}
\item Copy of the irrBullet wrapper which can be found at https://github.com/CollectiveTyranny/irrBullet
\item A C++ project set up that can compile code that uses Bullet, and the Irrlicht 3D engine
\item Intermediate knowledge of C++, and the Irrlicht Engine
\end{itemize}
Tutorials about Irrlicht can be found at http://irrlicht.sourceforge.net/docu/example001.html and for people who like video tutorials https://www.youtube.com/watch?v=brAaoNhK4Mk.
\end{frame}

%------------------------------------------------

\begin{frame}
\frametitle{About Bullet Physics}
\begin{itemize}
\item 3D physics engine, handles physics dynamics as well as collision
\item Agnostic of all 3D graphics engines, many bindings made for other languages
\item Very advanced and customizable, many settings are available for almost all objects
\item Very poor documentation, it is very difficult to find a comprehensive source of examples or tutorials; especially for those who are inexperienced
\item Under the MIT license, so it can be used for all purposes including commercial purposes
\end{itemize}
\end{frame}

%------------------------------------------------

\begin{frame}
\frametitle{Tools used to make this tutorial series}
The libraries used in this series are: Irrlicht and Bullet physics, it is implied that you are in an environment where you can build using both of these libraries.
\begin{itemize}
\item irrBullet - wrapper maintained by Brigham Keys and Nicolas Ortega
\item Irrlicht - 3D rendering and event handling
\item Bullet - Physics simulation
\item CMake - Cross platform build system
\item Emacs - Text editor
\item Doxygen - Documentation generation
\item gtk-recordmyscreen - Desktop recording
\end{itemize}
\end{frame}

%------------------------------------------------

\end{document} 
